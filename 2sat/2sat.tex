\section{Présentation du problème}
\subsection{Vocabulaire et notation}
On nomme une \emph{inconnue} un littéral $v$ pouvant évaluer à \emph{vrai} ou
\emph{faux}. Une \emph{clause} est une proposition de la forme $\vee_{i=1}^n v_i$.
Si on note $(v_1,\ldots v_n), n\in\mathbb{N}^*$ les inconnues, on note :
    \[ \forall i\in\seg{1}{n}, \left\{\begin{array}{l}
            V_i = v_i \\
            V_{-i} = \neg v_i
       \end{array}\right.
     \]

On a donc $\neg V_i = V_{-i}$.

Posons $E_n = \mathscr{P}(\seg{-n}{-1}\cup\seg{1}{n})$. $C_n = \{\vee_{i\in I} V_i | I\in E_n\}$
décrit alors l'ensemble des clauses possibles à $n$ inconnues. Soit $C\in C_n$
et $I\in E_n$ l'ensemble associé. On nomme \emph{ordre} de $C$ le cardinal de
$I$ : $o(C) = |I|$.

Posons $P_n = \mathscr{P}(E_n)$. $S_n = \{\wedge_{I\in P} (\vee_{i\in I} V_i) | P\in P_n\}$
décrit l'ensemble des formules du calcul propositionnel en forme causale à $n$
inconnues. Soit $S\in S_n$ et $P\in P_n$ l'ensemble associé. On nomme ordre de
$S$ l'ordre de la plus grande clause de la formule : $o(S) = \max_{I\in P} |I|$.

\subsection{Problème SAT}
Soit $n\in\mathbb{N}^*$. Soit $S\in S_n$. $S$ est dit \emph{satisfaisable} s'il
existe des valeurs pour les inconnues telles que $S$ vale \emph{vrai} lorsqu'on
évalue ses variables en les valeurs données.

On nomme problème SAT le problème de savoir si un problème $S$ quelconque est
satisfaisable. Le problème est décidable puisque qu'il suffit de tester toutes
les valeurs possibles des inconnues pour voir si une convient. La complexité
est alors exponentielle. L'objectif est de trouver des algorithmes plus
efficaces. Le problème SAT est cependant $NP$-complet \cite{cook71}, donc
sous l'hypothèse $P\neq NP$, il n'existe pas d'algorithme le résolvant dans le
cas général en temps polynomial. On s'intéresse donc à des cas particuliers.

On nomme problème $k$-SAT le problème de savoir si un problème $S$ d'ordre au
plus $k$, $k\in\mathbb{N}^*$, est satisfaisable. Cependant, on peut montrer
que l'on peut réduire toute formule $S$ d'ordre quelconque en une formule
équivalente d'ordre au plus $3$ en temps polynomial.
\demo{$NP$-complétude du problème $3$-SAT}
Donc le problème $3$-SAT est $NP$-complet. On s'intéressera donc au problème
$2$-SAT, résoluble en temps quasi-linéaire.
\idee{Chercher toutes les solutions d'un problème $2$-SAT}

\section{Algorithme de résolution}
On crée $G$ le graphe orienté dont les nœuds sont les variables et leur
négation (s'il y a $n$ variables, il y a donc $2n$ nœuds). On étudie chaque
clause pour former les arrêtes. Le problème étant d'ordre au plus $2$, chaque
clause est d'ordre au plus $2$. On crée les arrêtes de la façon suivante :
\begin{itemize}
 \item Pour chaque clause de la forme $x_i \vee x_j$, on ajoute les arrêtes
       $\neg x_i \rightarrow x_j$ et $\neg x_j \rightarrow x_i$.
 \item Pour chaque clause de la forme $x_i$, on la considère équivalente à
       $x_i \vee x_i$, donc on ajoute $\neg x_i \rightarrow x_i$ au graphe (ce
       qui revient à dire que $x_i$ doit être vrai).
\end{itemize}

On calcul les composantes fortement connexes du graphe ainsi déterminé.

Le problème est satisfaisable si et seulement si chaque variable est dans une
composante connexe différente de sa négation.

\subsection{Preuve}
\subsubsection{Sens direct}
On procède par la contraposée. On suppose qu'il existe une variable qui soit
dans la même composante connexe que sa négation. On la note $x$. Le graphe
étant un graphe d'implication, on a alors $x \equival \neg x$. $x$ ne peut
vérifier cette condition, donc le problème n'est pas satisfaisable.

\subsubsection{Sens réciproque}
\tref{De qui vient l'idée de la démonstration}
On suppose que tout $x$ est dans une composante fortement connexe différente de
celle dans laquelle est $\neg x$. On va procéder à une preuve constructive, en
donnant un algorithme qui donnera une solution au problème.

Quelques notations d'abord. On note $G$ le graph orienté du problème, $S(G)$ le
graphe de ses composantes fortement connexes, qui est nécessairement acyclique.
De plus, on note $G^r$ le graphe orienté renversé et $\neg G$ le graph dont on
prend la négation de chaque nœud. On peut étendre cette notation : $\neg S(G)$
est le graphe dont on a pris la négation de chaque élément de chaque nœud.
Enfin, on note $N(G)$ l'ensemble des nœuds du graphe et $E(G) \in N(G)^2$ les
arrêtes.

Quelques résultats préliminaires, rapides à démontrer :
\demo{Résultat préliminaires de l'algorithme du $2$-SAT}
\begin{itemize}
 \item $\neg G^r = G$
 \item $\neg S(G) = S(\neg G)$
 \item $S(G)^r = S(G^r)$
 \item On en déduit $\neg S(G)^r = S(G)$
\end{itemize}

L'objectif est de marquer chaque composante fortement connexe avec vrai ou faux
de façon que si $x \rightarrow y$, alors $x$ faux ou $x$ et $y$ vrai. De plus,
il faut que $x$ soit marqué vrai si et seulement si $\neg x$ est marqué faux.
En applicant ces valeurs aux variables des composantes fortement connexes
(ce qui est alors possible), on a une solution.
\demo{Conditions pour que le coloriage de $S(G)$ soit solution}

On note $(S_1,\ldots S_p)$ les composantes fortement connexes de $G$ triées par
ordre topologique inverse :
    \[ \forall (i,j)\in\seg{1}{p}^2, S_i \rightarrow S_j \implies j \leq i \]

De plus, on a :
    \[ \forall i\in\seg{1}{p}, \exists j\in\seg{1}{p}\backslash\{i\}: \neg S_i = S_j \]

Le $i\neq j$ est lié à l'hypothèse de l'implication.

On applique alors l'algorithme suivant : on parcourt les composantes fortement
connexes dans l'ordre topologique inverse. Si la composante $x$ est déjà
marquée, on passe. Sinon, on marque $S$ vraie et $\neg S$ fausse.

À l'issu de cet algorithme, la condition $S$ marquée vraie si et seulement si
$\neg S$ marquée fausse est immédiatement vérifiée. De plus, $S_i$ est marquée
vraie si et seulement si $i < j$, avec $S_j = \neg S_i$ (marquée vraie
seulement si elle est rencontrée avant sa négation).

Supposons par l'absurde que l'on puisse trouve $(i,j)\in\seg{1}{p}^2$ tels que
$S_i\rightarrow S_j$, $S_i$ marqué vrai et $S_j$ marqué faux. On a donc
$j \neq i$. Notons aussi $(k,l)\in\seg{1}{p}^2$ tels que $S_k = \neg S_i$ et
$S_l = \neg S_j$. On a alors $i < k$ et $l < j$ (*) d'après la remarque
précédente. Comme $\neg S(G)^r = S(G)$, on a aussi $\neg S_j\rightarrow \neg S_i$,
soit $S_l\rightarrow S_k$. Or les $(S_i)$ sont classés par ordre topologique
inverse, d'où $k < l$ (**) et $j < i$ (***). Or (*) et (**) donnent $i < j$ :
contradiction avec (***).

Le marquage ainsi trouvé vérifie les conditions : il donne bien un solution,
donc le problème admet en effet une solution.


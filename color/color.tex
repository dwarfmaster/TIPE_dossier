\section{$NP$-complétude du $3$-coloriage de graphe.}
Plusieurs articles ont montré qu'il est possible, à partir d'un problème $3$-SAT,
de créer un graphe dont l'existence d'un $3$-coloriage est équivalente à la
satisfaisabilité du $3$-SAT considéré \cite{CMSC451}. La taille du graphe créé
étant polynomial en la taille du problème SAT, déterminer l'existence d'un
$3$-coloriage pour tout graphe en temps polynomial revient à résoudre le
$3$-SAT en temps polynomial : le $3$-coloriage est $NP$-complet.

\section{Graphe parfait}
Les graphes parfaits sont une catégorie de graphe très large pour lequel la
recherche du nombre chromatique se fait en temps polynomial \cite{GrParfaitTrotignon}.
De plus, leur reconnaissance se fait aussi en temps polynomial \cite{GrParfaitENSLyon}.
Un façon de résoudre certains $3$-SAT serait de créer le graphe selon la méthode
précédente, tester si il est parfait et si c'est le cas, calculer son nombre
chromatique. Bien que ces méthodes soient polynomiales, l'algorithme est
complexe, il est donc intéressant d'estimer la probabilité qu'un $3$-SAT donne
un graphe parfait.

En utilisant la caractérisation du théorème fort des graphes parfaits, on se
rend compte qu'aucune des méthodes usuelles pour obtenir un graphe ne donne
de graphe parfait. \demo{Non perfection des graphes usuels d'un $3$-SAT}

\idee{Trouver un graphe de $3$-SAT qui ait une chance d'être parfait}


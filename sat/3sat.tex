\section{Problème $3$-SAT}
Le problème $3$-SAT est donc un problème où chaque clause fait intervenir au
plus trois variables. Au contraire du $2$-SAT, ce problème n'est pas
significativement plus simple que le SAT puisqu'il est équivalent à ce derniers
au sens en terme de réduction polynomiale : il est aussi $NP$-complet.

\subsection{Preuve de la $NP$-difficulté du problème $3$-SAT}
On va montrer que tout problème SAT est réductible en un temps polynomiale à
un problème 3-SAT.

On considère la clause $v_1 \vee \ldots \vee v_n$ pour $n \geq 3$. Montrons que
cette clause est satisfaisable si et seulement si :
\[ (v_1 \vee v_2 \vee x_2) \wedge (\neg x_2 \vee v_3 \vee x_3) \wedge \ldots
\wedge (\neg x_{n-3} \vee v_{n-2} \vee x_{n-2}) \wedge (\neg x_{n-2} \vee v_{n-1} \vee v_n) \]
est satisfaisable. On nomme $S_1$ la clause étudiée et $S_2$ le système à
$2n - 3$ variables.

\tdoc{Améliorer l'intégration des vrai/faux en mode math}
Supposons que $S_1$ soit satisfaisable. Notons $(v_1,\ldots v_n)$ une
solution. Comme toute permutation des solutions est aussi solution, on peut
considérer que $v_1 \vee v_2$ est vrai. Posons alors
$\forall i\in\seg{2}{n-2}, x_i \leftarrow \text{faux}$. Dans ce cas, on a
$v_1 \vee v_2 \vee x_2 \rightarrow \text{vrai}$. De plus, pour $i\in\seg{2}{n-3}$,
on a $\neg x_i \vee v_{i+1} \vee x_{i+1} \rightarrow \text{vrai}$. Enfin, comme
$\neg x_{n-2} \rightarrow \text{vrai}$, on a
$\neg x_{n-2} \vee v_{n-1} \vee v_n \rightarrow \text{vrai}$. On a donc trouvé
une solution du problème $S_2$, il est donc satisfaisable.

Supposons maintenant que $S_2$ soit satisfaisable. Notons $(v_1,\ldots,v_n,x_2,\ldots,x_{n-2})$
une solution. Par l'absurde, supposons que $\forall i\in\seg{1}{n}, v_i \rightarrow \text{faux}$.
De plus, chaque clause du problème $S_2$ doit évaluer à vrai pour que $S_2$
évalue à vrai. On a donc $v_1\vee v_2\vee x_2 \rightarrow x_2$ donc
$x_2 \rightarrow \text{vrai}$. Par une récurrence finie immédiate, on a donc
que $\forall i\in\seg{2}{n-2}, x_i \rightarrow\text{vrai}$. Or dans ce cas,
$\neg x_{n-2}\vee v_{n-1}\vee v_n \rightarrow\text{faux}$, d'où un
contradiction. On au moins un $v_i$ est vrai, d'où $S_1$ est satisfaisable.

En prenant un problème $S$ d'ordre quelconque (mais fini) à $n$ inconnues, on
peut le réduire en un système d'ordre au plus $3$ à $2n-3$ inconnues. Le
problème $3$-SAT est donc $NP$-complet.


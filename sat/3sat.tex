\section{Problème $3$-SAT}
Le problème $3$-SAT est donc un problème où chaque clause fait intervenir au
plus trois variables. Au contraire du $2$-SAT, ce problème n'est pas
significativement plus simple que le SAT puisqu'il est équivalent à ce derniers
au sens en terme de réduction polynomiale : il est aussi $NP$-complet.

\subsection{Preuve de la $NP$-difficulté du problème $3$-SAT}
On va montrer que tout problème SAT est réductible en un temps polynomiale à
un problème 3-SAT.

Posons $\phi(v_1,\dots v_n) = v_1 \vee \dots \vee v_n$ une clause de taille
$n$ avec $n\geq 3$ et :
\[ \psi(v_1, \dots, v_n, x_2, \dots x_{n-2})
        = (v_1 \vee v_2 \vee x_1)
   \wedge (\neg x_2 \vee v_3 \vee x_3) \wedge \ldots
   \wedge (\neg x_{n-3} \vee v_{n-2} \vee x_{n-2})
   \wedge (\neg x_{n-2} \vee v_{n-1} \vee v_n) \]
\tdoc{Améliorer la découpe des longues expressions mathématiques}

$\psi$ est une formule logique sous forme conjonctive avec des clauses de
taille $3$ et $2n-3$ variables. On va montrer que $\psi \cosat \phi$.

Supposons que $\phi$ soit satisfaisable. Notons $(v_1,\ldots v_n)$ une
solution. Comme toute permutation des solutions est aussi solution, on peut
considérer que $v_1 \vee v_2 \equiv \top$. Alors, en considérant
$\forall i\in\seg{2}{n-2}, x_i \equiv \bot$, on a
$v_1 \vee v_2 \vee x_2 \equiv \top$. De plus, pour $i\in\seg{2}{n-3}$, on a
$\neg x_i \vee v_{i+1} \vee x_{i+1} \equiv \top$. Enfin, comme
$\neg x_{n-2} \equiv \top$, on a
$\neg x_{n-2} \vee v_{n-1} \vee v_n \equiv \top$. On a donc trouvé une
assignation satisfaisant $\psi$, qui est donc satisfaisable.

Supposons maintenant que $\psi$ soit satisfaisable. Notons
$(v_1,\ldots,v_n,x_2,\ldots,x_{n-2})$ une solution. Par l'absurde, supposons
que $\forall i\in\seg{1}{n}, v_i \equiv \bot$. De plus, chaque clause du
problème $\psi$ doit évaluer à vrai pour que $\psi$ évalue à vrai. On a donc
$v_1\vee v_2\vee x_2 \equiv x_2$ donc $x_2 \equiv \top$. Par une récurrence
finie immédiate, on a donc que $\forall i\in\seg{2}{n-2}, x_i \equiv\top$. Or
dans ce cas, $\neg x_{n-2}\vee v_{n-1}\vee v_n \equiv\bot$, d'où une
contradiction. On au moins un $v_i$ est vrai, d'où $\phi$ est satisfaisable.

En prenant un problème d'ordre quelconque (mais fini) à $n$ inconnues, on
peut le réduire en un système d'ordre au plus $3$ à $2n-3$ inconnues. Le
problème $3$-SAT est donc $NP$-complet.


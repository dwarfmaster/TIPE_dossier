\section{Présentation du problème}
Le problème SAT est un problème de décision étudiant la satisfiabilité d'une
formule booléenne, à savoir l'existence d'une assignation telle que la formule
booléenne soit vraie.

Autrement dit, en notant $\phi$ la formule considérée et $\bot$ la formule
fausse, on cherche à déterminer si $\phi\equiv\bot$ ou non.

Si $\phi\equiv\bot$, on dit que la formule est \texttt{UNSAT}, sinon on dit
qu'elle est \texttt{SAT} ou satisfiable.

\subsection{Notations}
On note $\top$ la formule toujours vraie, aussi appelée tautologie, et $\bot$
la formule toujours fausse.

On note $\vee$ l'opérateur \textit{ou}, aussi appelé disjonctif, et $\wedge$
l'opérateur \textit{et}, aussi appelé conjonctif. La négation est notée $\neg$.

L'équivalence de deux formules booléennes est notée $\equiv$. On a alors
$\top\equiv\neg\bot$.

Une \textit{formule logique} est une proposition qui ne fait intervenir que les
symboles $\vee$, $\wedge$, $\neg$, $\top$, $\bot$ et un certains nombre de
variables dont elle dépend.

Soit $\phi$ une formule logique. On note $\sat(\phi)$ la proposition affirmant
que $\phi$ est satisfiable et $\unsat(\phi)$ la proposition affirmant que $\phi$
ne l'est pas. Selon les définitions ci-dessus, on a alors :
\[ \unsat(\phi) \equiv \neg \sat(\phi) \equiv (\phi \equiv \bot) \]

On dit que deux formules $\phi$ et $\psi$ sont \textit{cosatifiables} si
$\sat(\phi)\equiv\sat(\psi)$. On note alors $\phi\cosat\psi$. $\cosat$ définit
une relation d'équivalence sur les formules logiques.

\subsection{Forme conjonctive}
Soit $\phi(x_1, \dots x_n)$ une formule logique dépendant de $n$ variables.
$\phi$ peut s'écrire sous forme conjonctive, c'est à dire comme une disjonction
de conjonction des variables (ou de leur négation).

Par exemple, $x_1 \wedge (x_2 \vee (x_3 \wedge x_4))$ peut être écrit de façon
équivalente $x_1 \wedge (x_2 \vee x_3) \wedge (x_2 \vee x_4)$.

Il est toujours possible de transformer une formule logique en une formule
équivalente sous forme conjonctive (abrégée CNF), mais la taille de
l'expression peut augmenter exponentiellement. \idee{Donner un algorithme}

On dit que la formule est écrite comme étant un disjonction de \textit{clauses}.
On nomme cardinal ou taille d'une clause le nombre de variables distinctes
qu'elle fait intervenir.

\subsection{Complexité}
Sous sa forme générale, le problème SAT est $NP$-complet. Ce résultat a été
montré par Cook en 1971 \cite{cook71}, en encodant la succession d'états
possibles d'une machine de Turing non déterministe dans un problème SAT dont
le nombre de variable et de clause polynomial en le temps de terminaison de
la machine considérée, puis en montrant que la satisfiabilité de la formule
ainsi créé est équivalente à l'appartenance d'un mot au language codé par le
machine de Turing.

Un classe particulière de problèmes SAT est souvent étudiée : les problèmes
$k$-SAT, avec $k\in\mathbb{N}^*$. Dans ces problèmes, la formule logique doit
être donnée sous forme conjonctive, avec la condition supplémentaire que la
taille maximale d'une clause doit être inférieure à $k$. Le cas $k=1$ n'est pas
intéressant puisque résoluble immédiatement. Les deux cas étudiés sont $k=2$,
résoluble en temps polynomial, et le cas $k=3$, $NP$-complet.

\section{Problème $3$-SAT}
Le problème $3$-SAT est donc un problème où chaque clause fait intervenir au
plus trois variables. Au contraire du $2$-SAT, ce problème n'est pas
significativement plus simple que le SAT puisqu'il est équivalent à ce derniers
au sens en terme de réduction polynomiale : il est aussi $NP$-complet.

\subsection{Preuve de la $NP$-difficulté du problème $3$-SAT}
On va montrer que tout problème SAT est réductible en un temps polynomiale à
un problème 3-SAT.

Posons $\phi(v_1,\dots v_n) = v_1 \vee \dots \vee v_n$ une clause de taille
$n$ avec $n\geq 3$ et :
\[ \psi(v_1, \dots, v_n, x_2, \dots x_{n-2})
        = (v_1 \vee v_2 \vee x_1)
   \wedge (\neg x_2 \vee v_3 \vee x_3) \wedge \ldots
   \wedge (\neg x_{n-3} \vee v_{n-2} \vee x_{n-2})
   \wedge (\neg x_{n-2} \vee v_{n-1} \vee v_n) \]
\tdoc{Améliorer la découpe des longues expressions mathématiques}

$\psi$ est une formule logique sous forme conjonctive avec des clauses de
taille $3$ et $2n-3$ variables. On va montrer que $\psi \cosat \phi$.

Supposons que $\phi$ soit satisfaisable. Notons $(v_1,\ldots v_n)$ une
solution. Comme toute permutation des solutions est aussi solution, on peut
considérer que $v_1 \vee v_2 \equiv \top$. Alors, en considérant
$\forall i\in\seg{2}{n-2}, x_i \equiv \bot$, on a
$v_1 \vee v_2 \vee x_2 \equiv \top$. De plus, pour $i\in\seg{2}{n-3}$, on a
$\neg x_i \vee v_{i+1} \vee x_{i+1} \equiv \top$. Enfin, comme
$\neg x_{n-2} \equiv \top$, on a
$\neg x_{n-2} \vee v_{n-1} \vee v_n \equiv \top$. On a donc trouvé une
assignation satisfaisant $\psi$, qui est donc satisfaisable.

Supposons maintenant que $\psi$ soit satisfaisable. Notons
$(v_1,\ldots,v_n,x_2,\ldots,x_{n-2})$ une solution. Par l'absurde, supposons
que $\forall i\in\seg{1}{n}, v_i \equiv \bot$. De plus, chaque clause du
problème $\psi$ doit évaluer à vrai pour que $\psi$ évalue à vrai. On a donc
$v_1\vee v_2\vee x_2 \equiv x_2$ donc $x_2 \equiv \top$. Par une récurrence
finie immédiate, on a donc que $\forall i\in\seg{2}{n-2}, x_i \equiv\top$. Or
dans ce cas, $\neg x_{n-2}\vee v_{n-1}\vee v_n \equiv\bot$, d'où une
contradiction. On au moins un $v_i$ est vrai, d'où $\phi$ est satisfaisable.

En prenant un problème d'ordre quelconque (mais fini) à $n$ inconnues, on
peut le réduire en un système d'ordre au plus $3$ à $2n-3$ inconnues. Le
problème $3$-SAT est donc $NP$-complet.


\section{Problème 2-SAT}
Le problème $2$-SAT est un cas particulier du problème SAT qui a l'avantage
d'être résoluble en temps polynomial. \idee{Complexité exacte de l'algorithme}
Nous présentons ici un algorithme permettant de résoudre tout problème $2$-SAT.

\subsection{Algorithme de résolution}
On crée $G$ le graphe orienté dont les nœuds sont les variables et leur
négation (s'il y a $n$ variables, il y a donc $2n$ nœuds). On étudie chaque
clause pour former les arrêtes. Le problème étant d'ordre au plus $2$, chaque
clause est d'ordre au plus $2$. On crée les arrêtes de la façon suivante :
\begin{itemize}
 \item Pour chaque clause de la forme $x_i \vee x_j$, on ajoute les arrêtes
       $\neg x_i \rightarrow x_j$ et $\neg x_j \rightarrow x_i$.
 \item Pour chaque clause de la forme $x_i$, on la considère équivalente à
       $x_i \vee x_i$, donc on ajoute $\neg x_i \rightarrow x_i$ au graphe (ce
       qui revient à dire que $x_i$ doit être vrai).
\end{itemize}

On calcul les composantes fortement connexes du graphe ainsi déterminé. Ce
calcul peut se faire linéairement en le nombre de noeuds plus le nombres
d'arêtes de graphe en utilisant un algorithme comme l'algorithme de Tarjan. Il
a en plus donne les composantes connexes triées topologiquement, ce qui
peut être utile pour donner une solution explicite (voire la preuve pour
les détails).

Le problème est satisfaisable si et seulement si chaque variable est dans une
composante connexe différente de sa négation.

\subsection{Preuve}
\subsubsection{Sens direct}
On procède par la contraposée. On suppose qu'il existe une variable qui soit
dans la même composante connexe que sa négation. On la note $x$. Le graphe
étant un graphe d'implication, on a alors $x \equival \neg x$. $x$ ne peut
vérifier cette condition, donc le problème n'est pas satisfaisable.

\subsubsection{Sens réciproque}
\tref{De qui vient l'idée de la démonstration}
On suppose que tout $x$ est dans une composante fortement connexe différente de
celle dans laquelle est $\neg x$. On va procéder à une preuve constructive, en
donnant un algorithme qui donnera une solution au problème.

Quelques notations d'abord. On note $G$ le graph orienté du problème, $S(G)$ le
graphe de ses composantes fortement connexes, qui est nécessairement acyclique.
De plus, on note $G^r$ le graphe orienté renversé et $\neg G$ le graph dont on
prend la négation de chaque nœud. On peut étendre cette notation : $\neg S(G)$
est le graphe dont on a pris la négation de chaque élément de chaque nœud.
Enfin, on note $N(G)$ l'ensemble des nœuds du graphe et $E(G) \in N(G)^2$ les
arrêtes.

Quelques résultats préliminaires, rapides à démontrer :
\demo{Résultat préliminaires de l'algorithme du $2$-SAT}
\begin{itemize}
 \item $\neg G^r = G$
 \item $\neg S(G) = S(\neg G)$
 \item $S(G)^r = S(G^r)$
 \item On en déduit $\neg S(G)^r = S(G)$
\end{itemize}

L'objectif est de marquer chaque composante fortement connexe avec vrai ou faux
de façon que si $x \rightarrow y$, alors $x$ faux ou $x$ et $y$ vrai. De plus,
il faut que $x$ soit marqué vrai si et seulement si $\neg x$ est marqué faux.
En applicant ces valeurs aux variables des composantes fortement connexes
(ce qui est alors possible), on a une solution.
\demo{Conditions pour que le coloriage de $S(G)$ soit solution}

On note $(S_1,\ldots S_p)$ les composantes fortement connexes de $G$ triées par
ordre topologique inverse :
    \[ \forall (i,j)\in\seg{1}{p}^2, S_i \rightarrow S_j \implies j \leq i \]

De plus, on a :
    \[ \forall i\in\seg{1}{p}, \exists j\in\seg{1}{p}\backslash\{i\}: \neg S_i = S_j \]

Le $i\neq j$ est lié à l'hypothèse de l'implication.

On applique alors l'algorithme suivant : on parcourt les composantes fortement
connexes dans l'ordre topologique inverse. Si la composante $x$ est déjà
marquée, on passe. Sinon, on marque $S$ vraie et $\neg S$ fausse.

À l'issu de cet algorithme, la condition $S$ marquée vraie si et seulement si
$\neg S$ marquée fausse est immédiatement vérifiée. De plus, $S_i$ est marquée
vraie si et seulement si $i < j$, avec $S_j = \neg S_i$ (marquée vraie
seulement si elle est rencontrée avant sa négation).

Supposons par l'absurde que l'on puisse trouve $(i,j)\in\seg{1}{p}^2$ tels que
$S_i\rightarrow S_j$, $S_i$ marqué vrai et $S_j$ marqué faux. On a donc
$j \neq i$. Notons aussi $(k,l)\in\seg{1}{p}^2$ tels que $S_k = \neg S_i$ et
$S_l = \neg S_j$. On a alors $i < k$ et $l < j$ (*) d'après la remarque
précédente. Comme $\neg S(G)^r = S(G)$, on a aussi $\neg S_j\rightarrow \neg S_i$,
soit $S_l\rightarrow S_k$. Or les $(S_i)$ sont classés par ordre topologique
inverse, d'où $k < l$ (**) et $j < i$ (***). Or (*) et (**) donnent $i < j$ :
contradiction avec (***).

Le marquage ainsi trouvé vérifie les conditions : il donne bien un solution,
donc le problème admet en effet une solution.




\section{Problème 2-SAT}
Le problème $2$-SAT est un cas particulier du problème SAT qui a l'avantage
d'être résoluble en temps polynomial. \idee{Complexité exacte de l'algorithme}
Nous présentons ici un algorithme permettant de résoudre tout problème $2$-SAT.

\subsection{Algorithme de résolution}
On crée $G$ le graphe orienté dont les nœuds sont les variables et leur
négation (s'il y a $n$ variables, il y a donc $2n$ nœuds). On étudie chaque
clause pour former les arrêtes. Le problème étant d'ordre au plus $2$, chaque
clause est d'ordre au plus $2$. On crée les arrêtes de la façon suivante :
\begin{itemize}
 \item Pour chaque clause de la forme $x_i \vee x_j$, on ajoute les arrêtes
       $\neg x_i \rightarrow x_j$ et $\neg x_j \rightarrow x_i$.
 \item Pour chaque clause de la forme $x_i$, on la considère équivalente à
       $x_i \vee x_i$, donc on ajoute $\neg x_i \rightarrow x_i$ au graphe (ce
       qui revient à dire que $x_i$ doit être vrai).
\end{itemize}

On calcule les composantes fortement connexes du graphe ainsi déterminé. Ce
calcul peut se faire linéairement en le nombre de noeuds plus le nombres
d'arêtes de graphe en utilisant un algorithme comme l'algorithme de Tarjan. Il
donne en plus les composantes connexes triées topologiquement, ce qui peut être
utile pour donner une solution explicite (voire la preuve pour les détails).

Le problème est satisfaisable si et seulement si chaque variable est dans une
composante connexe différente de sa négation.

\subsection{Preuve}
L'idée sous-jacente de la preuve est que :
\[ x_i\vee x_j \equiv (\neg x_i \implies x_j) \wedge (\neg x_j \implies x_i) \]

\subsubsection{Sens direct}
On procède par la contraposée. On suppose qu'il existe une variable qui soit
dans la même composante connexe que sa négation. On la note $x$. Le graphe
étant un graphe d'implication, on a alors $x \equiv \neg x$. $x$ ne peut
vérifier cette condition, donc le problème n'est pas satisfaisable.

\subsubsection{Sens réciproque}
On suppose que tout $x$ est dans une composante fortement connexe différente de
celle dans laquelle est $\neg x$. On va procéder à une preuve constructive, en
donnant un algorithme qui donnera une solution au problème.

Quelques notations d'abord. On note $G$ le graph orienté du problème, $S(G)$ le
graphe de ses composantes fortement connexes, qui est nécessairement acyclique.
De plus, on note $G^r$ le graphe orienté renversé et $\neg G$ le graph dont on
prend la négation de chaque nœud. On peut étendre cette notation : $\neg S(G)$
est le graphe dont on a pris la négation de chaque élément de chaque nœud.
Enfin, on note $N(G)$ l'ensemble des nœuds du graphe et $E(G) \in N(G)^2$ les
arrêtes.

Quelques résultats préliminaires, rapides à démontrer :
\demo{Résultat préliminaires de l'algorithme du $2$-SAT}
\begin{itemize}
 \item $\neg G^r = G$
 \item $\neg S(G) = S(\neg G)$
 \item $S(G)^r = S(G^r)$
 \item On en déduit $\neg S(G)^r = S(G)$
\end{itemize}

L'objectif est de marquer chaque composante fortement connexe avec vrai ou faux
de façon que si $x \rightarrow y$, alors $x$ faux ou $x$ et $y$ vrai. De plus,
il faut que $x$ soit marqué vrai si et seulement si $\neg x$ est marqué faux.
En applicant ces valeurs aux variables des composantes fortement connexes
(ce qui est alors possible), on a une solution d'après l'équivalence présentée
en début de preuve.

On note $(S_1,\ldots S_p)$ les composantes fortement connexes de $G$ triées par
ordre topologique inverse :
    \[ \forall (i,j)\in\seg{1}{p}^2, S_i \rightarrow S_j \implies j \leq i \]

De plus, on a :
    \[ \forall i\in\seg{1}{p}, \exists j\in\seg{1}{p}\backslash\{i\}: \neg S_i = S_j \]

Le $i\neq j$ est lié à l'hypothèse de l'implication.

On applique alors l'algorithme suivant : on parcourt les composantes fortement
connexes dans l'ordre topologique inverse. Si la composante $x$ est déjà
marquée, on passe. Sinon, on marque $S$ vraie et $\neg S$ fausse.

À l'issu de cet algorithme, la condition $S$ marquée vraie si et seulement si
$\neg S$ marquée fausse est immédiatement vérifiée. De plus, $S_i$ est marquée
vraie si et seulement si $i < j$, avec $S_j = \neg S_i$ (marquée vraie
seulement si elle est rencontrée avant sa négation).

Supposons par l'absurde que l'on puisse trouve $(i,j)\in\seg{1}{p}^2$ tels que
$S_i\rightarrow S_j$, $S_i$ marqué vrai et $S_j$ marqué faux. On a donc
$j \neq i$. Notons aussi $(k,l)\in\seg{1}{p}^2$ tels que $S_k = \neg S_i$ et
$S_l = \neg S_j$. On a alors $i < k$ et $l < j$ $(1)$ d'après la remarque
précédente. Comme $\neg S(G)^r = S(G)$, on a aussi $\neg S_j\rightarrow \neg S_i$,
soit $S_l\rightarrow S_k$. Or les $(S_i)$ sont classés par ordre topologique
inverse, d'où $k < l$ $(2)$ et $j < i$ $(3)$. Or $(1)$ et $(2)$ donnent
$i < j$ : contradiction avec $(3)$.

Le marquage ainsi trouvé vérifie les conditions : il donne bien une solution,
donc le problème admet en effet une solution.


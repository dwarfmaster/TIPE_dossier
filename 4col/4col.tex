\section{Équivalence avec produits vectoriels}
\subsection{Produits vectoriels}
Soient $(x_1, \ldots x_n)$ $n$ variables de $\mathbb{R}^3$. On note
$x_1 \times \ldots \times x_n$ leur produit vectoriel : le produit vectoriel
n'étant pas associatif, c'est juste une notation. On note $L$ et $R$ deux ordre
de calculs, c'est à dire des parenthésages suffisants pour avoir un calcul non
ambigu du produit. Résoudre $L(x_1, \ldots x_n) = R(x_1, \ldots x_n)$ revient à
chercher une solution à ces ordres de calculs en utilisant uniquement les
vecteurs de la base canonique.

\subsection{Équivalence}
Le théorème des quatre couleurs est équivalent à l'existence d'une solution
pour chacun des problème d'ordre de calcul de produit vectoriel. Pour les
preuves, voir \cite{osorio09} et \cite{4col}. Voir aussi \url{grail.cba.csuohio.edu/~somos/4ct.html}.
\idee{Trouver une équivalence du même style du 2-sat}

